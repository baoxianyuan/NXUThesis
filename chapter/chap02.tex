\chapter{格式要求}

\section{论文正文}

论文正文 是主体,一般由标题、文字叙述、图、表格和公式等部分构成[1] 。一般可包括理论分析、计算方法、实验装置和测试方法,经过整理加工的实验结果分析和讨论,与理论计算结果的比较以及本研究方法与已有研究方法的比较等,因学科性质不同可有所变化。

论文内容一般应由十个主要部分组成,依次为:1. 封面,2. 中文摘要,3. 英文摘要,4. 目录,5. 符号说明,6. 论文正文,7. 参考文献,8. 附录,9. 致谢,10. 攻读学位期间发表的相关学术成果。

以上各部分独立为一部分,每部分应从新的一页开始。

\section{字数要求}

\subsection{硕士论文要求}

学术型论文不少于30000字;专业型论文不少于30000字。也可根据各学科自定。

\subsection{博士论文要求}

不少于80000字。也可根据各学科自定。

\section{其他要求}

\subsection{页面设置}

页边距:上3cm,下3cm,左3cm,右2.5cm,装订线靠左0.5cm位置。

页眉:2.5cm。页脚:2.5cm。

页眉从摘要页开始到论文最后一页均需设置。页眉内容:左对齐为“宁夏大学博士学位论文”或“宁夏大学硕士学位论文”,右对齐为各章章名。页眉打印字号为5号宋体,页眉之下有一条下划线。

页码从摘要开始,前置部分(摘要,Abstract,目录等)用大写罗马数字(Ⅰ,Ⅱ,Ⅲ,……)连续编排。正文部分从引言(或绪论)的首页开始,作为第1页,并从右页起页,按阿拉伯数字(1,2,3,……)从1开始连续编排直到文末。页码必须统一标注在每页页脚中部,采用Times New Roman,小五号居中书写。页码数字两侧不要加“-”等修饰线。力求不出空白页,如有仍应以右页作为单页页码。

\subsection{字体}

英文与数字字体要求为Times New Roman。如果英文与数字夹杂出现在黑体中文中,则将英文与数字采用Times New Roman字体再加粗。

\section{本章小结}

本章介绍了…… 
