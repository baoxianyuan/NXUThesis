\chapter{宁夏大学学位论文书写规范}

研究生学位论文是研究生科学研究工作的全面总结, 是描述其研究成果、代表其研究水平的重要学术文献资料,是申请和授予相应学位的基本依据。学位论文撰写是研究生培养过程的重要环节和基本训练之一,必须按照确定的规范认真执行。指导教师应加强指导,严格把关。

论文撰写应符合国家及各专业部门制定的有关标准,符合汉语语法规范。

硕士和博士学位论文,除在字数、理论研究的深度及创造性成果等方面的要求不同外,对其撰写规范的要求基本一致。
为此,根据《中华人民共和国国家标准科学技术报告、学位论文和学术论文的编写格式》结合兄弟院校的具体作法,制定如下规定。

\section{学位论文版式、格式}

\begin{enumerate}
	\item 论文开本及版芯\\
	论文开本大小:210mm×297mm(A4纸)\\
	版芯要求:左边距:30mm,右边距:25mm,上边距:30mm,下边距:25mm,页眉边距:23mm,页脚边距:18mm
	\item 除外语类专业外,论文用中文撰写
	\item 标题:论文分三级标题\\
	一级标题:黑体,三号或16pt,段前、段后间距为1行\\
	二级标题:黑体,四号或14pt,段前、段后间距为1行\\
	三级标题:黑体,小四号或12pt,段前、段后间距为1行\\
	上述段前、段后间距可适当调节,以便于控制正文合适的换页位置
	\item 正文字体:正文采用五号宋体,行间距为18磅;图、表标题采用小五号黑体;表格中文字、图例说明采用小五号宋体;表注采用六号宋体
	\item 页眉、页脚文字均采用小五号宋体,页眉左侧为“宁夏大学博(硕)士学位论文”,右侧为一级标题名称;页眉下横线为上粗下细文武线(3磅);单面复印时页码排在页脚居中位置,双面复印时页码分别按左右侧排列,页码一律以“--1--”格式排在相应的位置
	\item 文中表格均采用标准表格形式(如三线表,可参照正式出版物中的表格形式)
	\item 文中所列图形应有所选择,照片不得直接粘贴,须经扫描后以图片形式插入
	\item 文中英文、罗马字符一般采用Time New Roman正体,按规定应采用斜体的采用斜体
	\item 博士学位论文8万字,硕士学位论文3万字,专业学位论文3万字左右。
\end{enumerate}

\section{学位论文的各组成部分与排列顺序}

学位论文,一般由封面、独创性声明及版权授权书、中文摘要、英文摘要、目录、插图和附表清单、主要符号表、引言(第一章)、正文、注释、结论(最后一章)、参考文献、附录、致谢和作者简历及论文发表情况等部分组成并按前后顺序排列。

1.封面:不同类型研究生,学位论文封面、书脊要求如下:

\begin{enumerate}[label =(\arabic*)]
	\item 学位论文题目应能概括论文的主要内容,切题、简洁,不超过26字,可分两行排列,中英文对照;
	\item 未经学位评定分委员会遴选且在研究生院备案的合作指导教师,不得在学位论文上署名;署名的合作指导教师人数不超过2人;
	\item 学科门类:哲学、经济学、法学、教育学、文学、历史学、理学、工学、农学、管理学;学位级别:硕士、博士;专业学位:教育硕士、工商管理硕士(MBA)、农业推广硕士、工程硕士。
	\item 专业名称、专业领域名称、研究方向应严格按照专业目录和培养方案填写;
	\item 分类号:按《中国图书资料分类法》要求填写,可自行上网查询;
	\item 密级:涉密论文,由院学位评定分委员会根据国家规定的密级范围和法定程序审查确定密级,并注明相应保密年限;不需保密的应填写“公开”。
	\item 日期:学位论文完成时间。
\end{enumerate}

学历硕士封面颜色为草绿色;博士学位论文封面颜色为湖水蓝色;专业学位硕士学位论文的封面均为浅黄色;以同等学力在职申请硕士学位论文和高校教师在职攻读硕士学位论文封面颜色为浅蓝色。

2.独创性声明和关于论文使用授权的说明(见附件二)附于学位论文摘要之前,需研究生和指导教师本人签字。

3.中文摘要(见附件三):硕士论文摘要的字数一般为500个左右,博士论文摘要的字数为800-1000个。内容包括研究工作目的、研究方法、所取得的结果和结论,应突出本论文的创造性成果或新见解,语言精炼。摘要应当具有独立性,即不阅读论文的全文,就能获得论文所能提供的主要信息。

为便于文献检索,应在论文摘要后另起一行注明本文的关键词3~5个,中间以“,”分隔。

4.英文摘要(见附件三):与中文摘要对应。

5.目录(见附件四-一与四-二):应是论文的提纲,也是论文组成部分的小标题。目录一般列至二级标题。

6.插图和附表清单:论文中如果图、表较多,可以分别列出清单列于目录页之后。图表的清单应有序号、图表名称和页码。

7.符号、标志、缩略词、计量单位、名词、术语等注释说明,可以集中列于图表的清单之后。

8.引言(第一章):在论文正文前。内容包括:该研究工作的实用价值和理论意义;国内外已有的文献综述;本研究要解决的问题。

9.正文:是学位论文的主体。写作内容可因研究课题的性质而不同,一般包括:理论分析、计算方法、实验装置和测试方法、对实验结果或调研结果的分析与讨论,本研究方法与已有研究方法的比较等方面。内容应简炼、重点突出,不要叙述专业方面的常识性内容。各章节之间应密切联系,形成一个整体。

10.注释:一律采用脚注方式,如“①”等,按照本学科国内外通行的范式,逐一注明本文引用或参考、借用的资料数据出处及他人的研究成果和观点,严禁掠人之美和抄袭剽窃。

11.结论(最后一章):结论应明确、简炼、完整、准确,要认真阐述自己的研究工作在本领域中的地位、作用以及自己新见解的意义。应当严格区分研究生的成果与导师的科研成果的界限。

如果不可能得出明确结论,也可以没有结论而进行必要的讨论。

12.参考文献:引用他人的成果必须标明出处。所有引用过的文献,应按引用的顺序编号排列。参考文献一律放在结论之后,不得放在各章之后。

13.附录:凡不宜放在论文正文中,但又与论文有关的研究过程或资料,如较为冗长的公式推导、重复性或者辅助性数据图表、计算程序及有关说明等,均应放入附录。

14.致谢:致谢对象限于对课题研究、学位论文完成等方面有较重要帮助的人员,落款处签名并注明日期(大写日期)。

15.个人简历及论文发表情况

个人简历包括个人姓名、性别、民族、出生年、籍贯,本科阶段入学时间、所在学校、所学专业,进入宁夏大学攻读学位时间、攻读专业。论文发表情况只登记已经正式发表者,格式同参考文献中论文格式。

\section{书写要求}

1.语言表述

论文应层次分明、数据可靠、文字简炼、说明透彻、推理严谨、立论正确,避免使用文学性质的带感情色彩的非学术性词语。论文中如出现非通用性的新名词、新术语、新概念,应作相应解释。

2.层次和标题

层次应清楚,标题应简明扼要,重点突出。

理工农类具体格式如下:

第一章   □□□□□(一级标题,居中,单列一行)

1.1   □□□□□(二级标题,左对齐,单列一行)

1.1.1   □□□□□(三级标题,左对齐,单列一行)
\newline
文史类具体格式如下:

第一章   □□□□□□□(一级标题,居中,单列一行)

第一节   □□□□□(二级标题,居中,单列一行)

一、□□□□(三级标题,首行缩进2字符,单列一行)

若有四五六级标题,可按如下格式编排:

(一)□□□□□(四级标题,首行缩进2字符)

1、□□□□□(五级标题,首行缩进2字符)

(1)□□□□□(六级标题,首行缩进2字符)

其它标题或需突出的重点,可用五号黑体(或加粗),可单列一行,也可放在段首。

3.篇眉和页码

从第一章开始书写篇眉,篇眉下为上粗下细文武线。页码从第一章开始按阿拉伯数字连续编排。第一章之前的页码用罗马数字单独编排。

4.图、表、公式等

图形要精选,要具有自明性,切忌与表及文字表述重复。图形坐标比例不宜过大,同一图形中不同曲线的图标应采用不同的形状和不同颜色的连线。图中术语、符号、单位等应与正文中表述一致。图序、标题、图例说明居中置于图的下方。

表中参数应标明量和单位。表序、标题居中置于表的上方。表注置于表的下方。\cite{zhangkun1994}

图、表应与说明文字相配合,图形不能跨页显示,表格一般放在同一页内显示。

公式一般居中对齐,公式编号用小括号括起,右对齐,其间不加线条。

文中的图、表、公式、附注等一律用阿拉伯数字按章节(或连续)编号,如图1-1,表2-2,公式(3-10)等。

5.参考文献

参考文献可顺序编码,也可按“著者-出版年”编码,也可以根据《中国高校自然科学学报编排规范》的要求书写参考文献,并按顺序编码,即按文中引用的顺序编码。作者姓名写至第三位,余者写“,等”或“,et al.”。

几种主要参考文献著录表的格式:

连续出版物:序号 作者. 文题. 刊名,年,卷号(期号):起~止页码
专(译)著:序号 作者. 书名(,译者). 出版地:出版者,出版年,起~止页码
论  文  集:序号 作者. 文题. 见(in):编者,编(eds). 文集名. 出版地:出版者,出版年,起~止页码
学 位 论 文:序号 作者. 文题:[XX学位论文]. 授予单位所在地:授予单位,授予年
专       利:序号 申请者. 专利名. 国名,专利文献种类,专利号,出版日期
技 术 标 准:序号 发布单位. 技术标准代号. 技术标准名称. 出版地:出版者,出版日期

举例如下:

[1] 朱文学. 粮食干燥原理及品质分析. 北京:高等教育出版社,2001,57-108
[2] Dupont B. Bone marrow transplantation in severe combined immunodeficiency with an unrelated MLC compatible donor. In:White H J.,Smith R,eds. Proceedings of the Third Annual Meeting of the International Society for Experimental Hematology. Houston:International Society for Experimental Hematology, 1974. 44-46
[3] 欧阳忠. 中国股市及农业板块的弱市场有效性假设的分析和应用:[硕士学位论文]. 北京:中国农业大学,2002
[4] 姜锡洲. 一种温热外敷药制备方法. 中国专利,881056073,1980-07-26
[5] 中华人民共和国国家技术监督局. GB3100~3102. 中华人民共和国国家标准—量与单位. 北京:中国标准出版社,1994-11-01

引用古籍分三种情况:

1、古籍整理本:时代·著者(或编者).书名.整理者.出版地:出版者,出版年.
如:[5]宋·苏辙. 苏辙集. 陈宏天、高秀芳点校. 北京:中华书局,1990.
2、丛书影印本:时代·著者(或编者).书名.丛书名.出版地:出版者,出版年.
如:[6]宋·陆游. 渭南文集.文渊阁《四库全书》本.台北:商务印书馆,1986.
3、古籍原本:严格按照古籍收藏单位著录之内容进行标注,基本格式为:时代·著者(或编者).书名.版本(括注公元年).藏书地点.
如:[7]元·虞集. 道园类稿. 元代至正十四年(1352)金伯祥刻本. 国家图书馆.

6.量和单位
应严格执行GB3100~3102:93有关量和单位的规定(参阅《常用量和单位》.计量出版社,1996)。单位名称的书写,可采用国际通用符号,也可用中文名称,但全文应统一,不要两种混用。

\section{电子文档要求}

1.电子版学位论文应与印刷本内容一致。因特殊情况出现不一致时,必须给予说明。

2.电子版学位论文应集合为一个word电子文档。采用其它编辑器编辑的论文,请提交pdf格式文件。